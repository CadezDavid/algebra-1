\newcommand{\sheet}{7}
\documentclass{article}

\usepackage[english, german]{babel}
\usepackage{amsthm,amssymb,amsmath,mathrsfs}
\usepackage[shortlabels]{enumitem}
% \usepackage[tmargin=1.25in,bmargin=1.25in,lmargin=1.2in,rmargin=1.2in]{geometry}


\newcommand{\C}{\mathbb{C}}
\newcommand{\R}{\mathbb{R}}
\newcommand{\N}{\mathbb{N}}
\newcommand{\Q}{\mathbb{Q}}
\newcommand{\Z}{\mathbb{Z}}

\DeclareMathOperator{\id}{id}
\DeclareMathOperator{\im}{im}
\DeclareMathOperator{\GL}{GL}
\DeclareMathOperator{\sgn}{sgn}
\DeclareMathOperator{\Tor}{Tor}

\newenvironment{exercise}[1] {
  \vspace{0.5cm}
  \noindent \textbf{Exercise~{#1}.}
} {
  \vspace{0.5cm}
}
\newenvironment{claim} {
  \noindent \textbf{Claim.}
} {
}

\title{Algebra 1\\Exercise sheet \sheet}
\author{Solutions by: Eric Rudolph and David Čadež}

\date{\today}


\begin{document}

\maketitle

\begin{exercise}{1}
    \begin{enumerate}
        \item Lets first prove that it is well-defined. For any $\varphi \colon
            M \rightarrow N$ we
            use the universal property of $B \otimes_A M$
            \[
                \begin{tikzcd}
                    &B \times M \arrow{d} \arrow{r}{\tau} &N \\
                    &B \otimes_A M \arrow{ur}
                \end{tikzcd}
            \]
            where $\tau \colon B \times M \rightarrow N$ with $\tau(b, m) = b
            \varphi(m)$. It is obviously bilinear, so universal property gives
            us the map in the exercise. So it is well defined.

            We can construct the inverse to the map given in the exercise with
            \begin{align*}
                \Hom_B(B \otimes_A M, N) &\rightarrow \Hom_A(M, N) \\
                \psi &\mapsto (m \mapsto \psi(1 \otimes m))
            \end{align*}
            It is easy to check that their compositions are identities.

        \item It is well defined, because it comes from a bilinear map
            \begin{align*}
                M \times N &\rightarrow (M \otimes_A B) \otimes_B N \\
                (m, n) &\mapsto (m \otimes 1) \otimes n
            \end{align*}

            It is an isomorphism, because it has an inverse
            \begin{align*}
                (M \otimes_A B) \otimes_B N &\rightarrow M \otimes_A N \\
                (m \otimes b) \otimes n &\mapsto m \otimes (b n)
            \end{align*}
            It is easy to check that their compositions are identities.

        \item We can define $B = S^{-1} A$ and $M = S^{-1} M_1, N = S^{-1} M_2$.
            We just have to put this in previous part and use $M \otimes_A
            S^{-1} A = M$.
    \end{enumerate}
\end{exercise}

\begin{exercise}{2}
    \begin{enumerate}
        \item We have to show
            \begin{equation}
                M_\mathfrak{p} = 0 \Longleftrightarrow M \otimes_A
                k(\mathfrak{p}) = 0.
            \end{equation}

            Suppose $M_\mathfrak{p} = 0$. Then for every $x \in M$ and $q \in A
            \setminus \mathfrak{p}$ we have $\frac{x}{q} = \frac{0}{1}$ which
            means there exists $r \in A \setminus \mathfrak{p}$ such that $r x =
            0$.
            Pick now any $\sum_i n_i \otimes (b_i + \mathfrak{p}) \in M \otimes_A
            k(\mathfrak{p})$. For every $n_i$ we find $r_i \in A \setminus
            \mathfrak{p}$ as above. Since $r_i \notin \mathfrak{p}$, it is non
            zero in $A / \mathfrak{p}$ and thus invertible in $\Quot(A /
            \mathfrak{p})$. So we get
            \begin{align*}
                \sum_i n_i \otimes (b_i  + \mathfrak{p}) &= 
                \sum_i n_i \otimes \left( \frac{r_i + \mathfrak{p}}{r_i +
                \mathfrak{p}}(b_i  + \mathfrak{p}) \right) \\
                &= \sum_i n_i \otimes \left( r_i \frac{1 + \mathfrak{p}}{r_i +
                \mathfrak{p}}(b_i  + \mathfrak{p}) \right) \\
                &= \sum_i r_i n_i \otimes \left( \frac{1 + \mathfrak{p}}{r_i +
                \mathfrak{p}}(b_i  + \mathfrak{p}) \right) \\
                &= \sum_i r_i n_i \otimes \left( \frac{1 + \mathfrak{p}}{r_i +
                \mathfrak{p}}(b_i  + \mathfrak{p}) \right) \\
                &= \sum_i 0 \otimes \left( \frac{1 + \mathfrak{p}}{r_i +
                \mathfrak{p}}(b_i  + \mathfrak{p}) \right) \\
                &= 0
            \end{align*}

            Which proves one implication.

        \item By associativity of the tensor product we have $\supp(M \otimes_A
            N) \subseteq \supp(M) \cap \supp(N)$. Suppose now $M_p \not= 0$ and
            $N_p \not= 0$. We have to show ${(M \otimes_A N)}_p \not= 0$. Using
            properties of localizations we get $M_p \otimes_{A_p} N_p \not= 0$.
            We know $A_p$ is a local ring with unique maximal ideal $p A_p$.

            Not sure how to continue.

    \end{enumerate}
\end{exercise}

\begin{exercise}{4}
    Let
    \begin{equation}
        \varphi \colon (f, T) \rightarrow A[T]
    \end{equation}
    be the inclusion. Then
    \begin{equation}
        \id \otimes\,\varphi \colon (f, T) \otimes_{A[T]} (f, T) \rightarrow (f,
        T) \otimes_{A[T]} A[T]
    \end{equation}
    is not injective, because
    \begin{equation}
        (\id \otimes\,\varphi)(f \otimes T - T \otimes f) = f \otimes T - T \otimes f = fT \otimes
        1 - fT \otimes 1 = 0
    \end{equation}
    but $f \otimes T \not= T \otimes f \in (f, T) \otimes_{A[T]} (f, T)$.
    Module $(f, T)$ is not flat by proposition 1 in lecture notes 12.
\end{exercise}

\end{document}

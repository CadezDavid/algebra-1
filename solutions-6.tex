\newcommand{\sheet}{6}
\documentclass{article}

\usepackage[english, german]{babel}
\usepackage{amsthm,amssymb,amsmath,mathrsfs}
\usepackage[shortlabels]{enumitem}
\usepackage{tikz}
\usepackage{tikz-cd}

% \usepackage[tmargin=1.25in,bmargin=1.25in,lmargin=1.2in,rmargin=1.2in]{geometry}


\newcommand{\C}{\mathbb{C}}
\newcommand{\R}{\mathbb{R}}
\newcommand{\N}{\mathbb{N}}
\newcommand{\Q}{\mathbb{Q}}
\newcommand{\Z}{\mathbb{Z}}

\DeclareMathOperator{\id}{id}
\DeclareMathOperator{\im}{im}
\DeclareMathOperator{\GL}{GL}
\DeclareMathOperator{\sgn}{sgn}
\DeclareMathOperator{\Tor}{Tor}
\DeclareMathOperator{\Sym}{Sym}

\newenvironment{exercise}[1] {
  \vspace{0.5cm}
  \noindent \textbf{Exercise~{#1}.}
} {
  \vspace{0.5cm}
}
\newenvironment{claim} {
  \noindent \textbf{Claim.}
} {
}

\title{Algebra 1\\Exercise sheet \sheet}
\author{Solutions by: Eric Rudolph and David Čadež}

\date{\today}


\begin{document}

\maketitle

\begin{exercise}{1}
    We note that the map is defined on elementary tensors.

    Well defined:
    Suppose $\sum^n_{i=1} a_i \otimes b_i = 0$. Then $(\sum^n_{i=1} a_i \cdot
    b_i) (1 \otimes 1) = 0$.

    Injectivity:
    Take $\sum^n_{i=1} a_i \otimes b_i \in \mathfrak{a} \otimes_A \mathfrak{b}$.
    Suppose $\sum^n_{i=1} a_i \cdot b_i = 0$. Then we can simply observe that
    since $a \otimes b = 1 \otimes (a b)$ we get
    \begin{equation*}
        \sum^n_{i=1} a_i \otimes b_i = 
        \sum^n_{i=1} 1 \otimes (a_i \cdot b_i) = 1 \otimes (\sum^n_{i=1} a_i
        \cdot b_i) = 1 \otimes 0 = 0
    \end{equation*}
    which proves injectivity.

    Surjectivity:
    An element $\sum^n_{i=1} a_i \cdot b_i \in \mathfrak{a} \cdot \mathfrak{b}$
    is the image of an element $\sum^n_{i=1} a_i \otimes b_i \in \mathfrak{a}
    \otimes_A \mathfrak{b}$.
\end{exercise}

\begin{exercise}{2}
    \begin{enumerate}
        \item Let $M = (b_1, \ldots, b_k)$ be the generators. Pick any
            \begin{equation*}
                    \prod_{i \in I} m_i \otimes n_i \in \prod_{i \in I} M \otimes_A
                    N_i.
            \end{equation*}
            It is enough to only look at these kinds of elements, because they
            generate $\prod_{i \in I} M \otimes_A N_i$.

            Then write $m_i = \sum^k_{j = 1} a_{ij} b_j$ for every $i \in I$.
            Then just observe
            \begin{equation*}
                \prod_{i \in I} (\sum^k_{j = 1} a_{ij} b_j) \otimes n_i = 
                \sum^k_{j = 1} \prod_{i \in I} a_{ij} b_j \otimes n_i = 
                \sum^k_{j = 1} \left( b_j \otimes (\prod_{i \in I} a_{ij} n_i)
                \right)
            \end{equation*}
            which is the image of $\sum^k_{j = 1} \left( b_j \otimes (\prod_{i
            \in I} a_{ij} n_i) \right) \in M \otimes_A \prod_{i \in I} N_i$.
    \end{enumerate}
\end{exercise}

\begin{exercise}{3}
\end{exercise}

\begin{exercise}{4}
    Let $J$ be the ``inverse'' of $I$, so $I \otimes_A J \cong A$. During the
    lectures we've shown that tensoring preserves exact sequences and in the
    proof we've shown that tensoring preserves surjectivity.
\end{exercise}

\end{document}

\newcommand{\sheet}{5}
\documentclass{article}

\usepackage[english, german]{babel}
\usepackage{amsthm,amssymb,amsmath,mathrsfs}
\usepackage[shortlabels]{enumitem}
% \usepackage[tmargin=1.25in,bmargin=1.25in,lmargin=1.2in,rmargin=1.2in]{geometry}


\newcommand{\C}{\mathbb{C}}
\newcommand{\R}{\mathbb{R}}
\newcommand{\N}{\mathbb{N}}
\newcommand{\Q}{\mathbb{Q}}
\newcommand{\Z}{\mathbb{Z}}

\DeclareMathOperator{\id}{id}
\DeclareMathOperator{\im}{im}
\DeclareMathOperator{\GL}{GL}
\DeclareMathOperator{\sgn}{sgn}
\DeclareMathOperator{\Tor}{Tor}

\newenvironment{exercise}[1] {
  \vspace{0.5cm}
  \noindent \textbf{Exercise~{#1}.}
} {
  \vspace{0.5cm}
}
\newenvironment{claim} {
  \noindent \textbf{Claim.}
} {
}

\title{Algebra 1\\Exercise sheet \sheet}
\author{Solutions by: Eric Rudolph and David Čadež}

\date{\today}


\begin{document}

\maketitle

\begin{exercise}{1}
    % Pick an ascending chain of ideals in $A$
    % \begin{equation*}
    %     I_0 \subseteq I_1 \subseteq I_2 \subseteq \cdots.
    % \end{equation*}
    % For every $j = 1, \ldots, n$ define $p_j \colon A \rightarrow A / a_j$ and
    % then $p_j(I_i) \subseteq A / a_j$ and observe the chains
    % \begin{equation*}
    %     p_j(I_0) \subseteq p_j(I_1) \subseteq p_j(I_2) \subseteq \cdots.
    % \end{equation*}
    % Since they all terminate, pick $M$ to be the index when the longest one
    % terminates. Suppose there exists $a \in I_{M+1} \setminus I_M$. By
    % definition of $M$ we have $p_j(a) \in p_j(I_M)$ for every $j = 1, \ldots,
    % n$. So $a = b_1 + c_1 = \cdots = b_n + c_n$ where $b_i \in I_M$ and $c_i \in
    % a_i$.

    Define $\varphi \colon A \rightarrow \bigoplus^n_{i=1} A / a_i$.

    Lets show that it is injective. Pick $a \in A$ with $\varphi(a) = 0$. Then
    $a \in a_j$ for every $j = 1, \ldots, n$. Since intersection of these ideals
    is trivial, we get $a = 0$.

    So $A$ is isomorphic to the image $\varphi(A)$. The image is a submodule of
    the module $\bigoplus^n_{i=1} A / a_i$. The direct sum $\bigoplus^n_{i=1} A
    / a_i$ is noetherian because it is the sum of neotherian modules. And the
    submodule of a noetherian module is obviously also noetherian.
\end{exercise}


\end{document}

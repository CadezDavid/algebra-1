\newcommand{\sheet}{5}
\documentclass{article}

\usepackage[english, german]{babel}
\usepackage{amsthm,amssymb,amsmath,mathrsfs}
\usepackage[shortlabels]{enumitem}
% \usepackage[tmargin=1.25in,bmargin=1.25in,lmargin=1.2in,rmargin=1.2in]{geometry}


\newcommand{\C}{\mathbb{C}}
\newcommand{\R}{\mathbb{R}}
\newcommand{\N}{\mathbb{N}}
\newcommand{\Q}{\mathbb{Q}}
\newcommand{\Z}{\mathbb{Z}}

\DeclareMathOperator{\id}{id}
\DeclareMathOperator{\im}{im}
\DeclareMathOperator{\GL}{GL}
\DeclareMathOperator{\sgn}{sgn}
\DeclareMathOperator{\Tor}{Tor}

\newenvironment{exercise}[1] {
  \vspace{0.5cm}
  \noindent \textbf{Exercise~{#1}.}
} {
  \vspace{0.5cm}
}
\newenvironment{claim} {
  \noindent \textbf{Claim.}
} {
}

\title{Algebra 1\\Exercise sheet \sheet}
\author{Solutions by: Eric Rudolph and David Čadež}

\date{\today}


\begin{document}

\maketitle

\begin{exercise}{1}
    % Pick an ascending chain of ideals in $A$
    % \begin{equation*}
    %     I_0 \subseteq I_1 \subseteq I_2 \subseteq \cdots.
    % \end{equation*}
    % For every $j = 1, \ldots, n$ define $p_j \colon A \rightarrow A / a_j$ and
    % then $p_j(I_i) \subseteq A / a_j$ and observe the chains
    % \begin{equation*}
    %     p_j(I_0) \subseteq p_j(I_1) \subseteq p_j(I_2) \subseteq \cdots.
    % \end{equation*}
    % Since they all terminate, pick $M$ to be the index when the longest one
    % terminates. Suppose there exists $a \in I_{M+1} \setminus I_M$. By
    % definition of $M$ we have $p_j(a) \in p_j(I_M)$ for every $j = 1, \ldots,
    % n$. So $a = b_1 + c_1 = \cdots = b_n + c_n$ where $b_i \in I_M$ and $c_i \in
    % a_i$.

    Define $\varphi \colon A \rightarrow \bigoplus^n_{i=1} A / a_i$.

    Lets show that it is injective. Pick $a \in A$ with $\varphi(a) = 0$. Then
    $a \in a_j$ for every $j = 1, \ldots, n$. Since intersection of these ideals
    is trivial, we get $a = 0$.

    So $A$ is isomorphic to the image $\varphi(A)$. The image is a submodule of
    the module $\bigoplus^n_{i=1} A / a_i$. The direct sum $\bigoplus^n_{i=1} A
    / a_i$ is noetherian because it is the sum of neotherian modules. And the
    submodule of a noetherian module is obviously also noetherian.
\end{exercise}

\begin{exercise}{2}
\end{exercise}

\begin{exercise}{3}
    In this exercise $\otimes$ will be written instead of $\otimes_A$.
    \begin{enumerate}
        \item With the condition $\Phi((\ldots, 0, n_i \otimes m, 0, \ldots)) =
            (\ldots, 0, n_i, 0, \ldots) \otimes m$ and additivity of $\Phi$ we
            can define it on each summand in $\bigoplus_{i \in I} (N_i \otimes
            M)$ separately. Using universal property we get unique maps
            \begin{align*}
                N_i \otimes M &\rightarrow (\bigoplus_{i \in I} N_i) \otimes M
                \\
                n \otimes m &\mapsto (\ldots, 0, n, 0, \ldots) \otimes m.
            \end{align*}
            They define a unique map
            \begin{align*}
                \Phi \colon \bigoplus_{i \in I} (N_i \otimes M) &\rightarrow
                (\bigoplus_{i \in I} N_i) \otimes M \\
                \sum_{i \in I} (n_i \otimes m) &\mapsto (\sum_{i \in I} n_i)
                \otimes m.
            \end{align*}
            Note that the definition is given on elementary tensors. It then
            extends linearly to all elements of the sum of tensor products. By
            construction it also satisfies the given condition.

            Let us now construct the inverse. Using universal property of the
            tensor product $(\bigoplus_{i \in I} N_i) \otimes M$ on the map
            (it is clearly bilinear)
            \begin{align*}
                (\bigoplus_{i \in I} N_i) \times M &\rightarrow \bigoplus_{i \in
                I} (N_i \otimes M) \\
                (\sum_{i \in I} n_i, m) &\mapsto \sum_{i \in I} (n_i \otimes m).
            \end{align*}
            We get a unique map
            \begin{align*}
                (\bigoplus_{i \in I} N_i) \otimes M &\rightarrow \bigoplus_{i
                \in I} (N_i \otimes M) \\
                (\sum_{i \in I} n_i) \otimes m &\mapsto \sum_{i \in I} (n_i
                \otimes m).
            \end{align*}
            It is clearly an inverse of $\Phi$.

        \item The map
            \begin{align*}
                A / \mathfrak{a} \times M &\rightarrow M / \mathfrak{a}M \\
                (a + \mathfrak{a}, m) &\mapsto m a + \mathfrak{a}
            \end{align*}
            Lets check it is well defined. If $a_1 + \mathfrak{a} = a_2 +
            \mathfrak{a}$, then $m (a_1 - a_2) \in \mathfrak{a}$ and $m a_1 +
            \mathfrak{a} = m a_2 + \mathfrak{a}$.

            It is clearly also bilinear.

            So by universal property it gives a unique map
            \begin{align*}
                \varphi \colon A / \mathfrak{a} \otimes M &\rightarrow M /
                \mathfrak{a}M \\
                (a + \mathfrak{a}) \otimes m &\mapsto a m + \mathfrak{a}M.
            \end{align*}

            Let us show the injectivity and surjectivity of $\varphi$.

            Injectivity: Suppose $a m \in \mathfrak{a}M$. Then $a m = a_1 m_1$
            for some $a_1 \in \mathfrak{a}$ and $m_1 \in M$. Calculate
            \begin{align*}
                (a + \mathfrak{a}) \otimes m &= (a(1 + \mathfrak{a})) \otimes m
                \\
                &= (1 + \mathfrak{a}) \otimes am \\
                &= (1 + \mathfrak{a}) \otimes a_1 m_1 \\
                &= a_1 (1 + \mathfrak{a}) \otimes m_1 \\
                &= (a_1 + \mathfrak{a}) \otimes m_1 \\
                &= 0 \otimes m_1 \\
                &= 0.
            \end{align*}

            Surjectivity:
            Take any $m + \mathfrak{a}M \in M/\mathfrak{a}M$. Then $\varphi((1 +
            \mathfrak{a}) \otimes m) = m + \mathfrak{a}M$.

            We could also construct the inverse
            \begin{align*}
                \gamma \colon M/\mathfrak{a}M &\rightarrow A/\mathfrak{a}
                \otimes M \\
                m + \mathfrak{a}M &\mapsto (1 + \mathfrak{a}) \otimes m.
            \end{align*}
    \end{enumerate}
\end{exercise}

\begin{exercise}{4}
    \begin{enumerate}[i)]
        \item We can construct $A$-module $\Sym^2_A(M)$ explicitly as
            \begin{equation*}
                \Sym^2_A(M) = (M \otimes_A M) / K,
            \end{equation*}
            where $K = (\{ m \otimes n - n \otimes m \mid m, n \in M \})$ is a
            submodule of $M \otimes_A M$ generated by the set in parenthesis. With
            this we ``make $M \otimes_A M$ commutative''.

            Define also a map
            \begin{align*}
                \iota \colon M \times M &\rightarrow \Sym^2_A(M) \\
                (m, n) &\mapsto m \otimes n + K.
            \end{align*}
            It is bilinear by the definition of the tensor product. By the
            definition of $K$ we have $m \otimes n + K = n \otimes m + K$, so it
            is also symmetric.

            Let now $(-, -) \colon M \times M \rightarrow N$ be any symmetric
            bilinear map.
            \[
                \begin{tikzcd}
                    &M \times M \arrow{d}{i} \arrow{r}{(-, -)} &N \\
                    &M \otimes_A M \arrow{r}{j} \arrow{ru}{f} &\Sym^2_A(M)
                    \arrow{u}{\Phi}
                \end{tikzcd}
            \]
            First we use that $(-, -)$ is bilinear, which gives us unique $f
            \colon M \otimes_A M \rightarrow N$. Using $f \circ i = (-, -)$ and
            that $(-, -)$ is symmetric we calculate
            \begin{equation*}
                f(m \otimes n) = (m, n) = (n, m) = f(n \otimes m)
            \end{equation*}
            which gives $f(m \otimes n - n \otimes m) = 0$ and thus $K \subseteq
            \ker f$, where $K$ is as in the definition of $\Sym^2_A(M)$. Thus it
            factors through the quotient $\Sym^2_A(M)$ uniquely. We get a unique
            $\Phi \colon \Sym^2_A(M) \rightarrow N$ for which $(-, -) = \Phi
            \circ \iota$. Note that $j \circ i = \iota$, since they were both
            defined in the obvious way.

            We construct $\Lambda^2_A(M)$ similarly:
            \begin{equation*}
                \Lambda^2_A(M) = (M \otimes_A M) / L,
            \end{equation*}
            where $L = (\{ m \otimes m \mid m \in M\})$ is the submodule
            generated by these ``diagonal elements''. The map $M \times M
            \rightarrow \Lambda^2_A(M)$ is defined in the obvious way
            \begin{align*}
                \gamma \colon M \times M &\rightarrow \Lambda^2_A(M) \\
                (m, n) &\mapsto m \otimes n + L.
            \end{align*}
            It is bilinear and alternating by the definition of $L$.
            We prove that any alternating bilinear $(-, -) \colon M \times M
            \rightarrow N$ factors through $\Lambda^2_A(M)$ the exact same way
            as above.

        \item Let $\{a_1, \ldots, a_n\}$ be the basis of the free module
            $A^n$. First construct the basis of $A^n \otimes_A A^n$. It is as we
            would expect $S = \{ a_i \otimes a_j \mid i, j \in \{1, \ldots,
            n\}\}$. It is clear if we observe where isomorphism below sends the
            basis elements
            \begin{equation*}
                (\bigoplus^n_{i=1} A) \otimes_A (\bigoplus^n_{j=1} A) \cong
                \bigoplus^n_{i=1} (A \otimes_A \bigoplus^n_{j=1} A) \cong
                \bigoplus^n_{i=1} \bigoplus^n_{j=1} A.
            \end{equation*}

            So $A^n \otimes_A A^n$ are again free modules with the basis $S$.
            Since quotients does not preserve the property of being free, we
            have to change this basis first. But quotienting by a submodule
            generated by some subset of the basis does preserve the property of
            being free. Define maps
            \begin{align*}
                \alpha \colon A^n \otimes_A A^n &\rightarrow A^n \otimes_A A^n
                \\
                a_i \otimes a_j &\mapsto 
                \begin{cases} 
                    a_i \otimes a_j - a_j \otimes a_i & i < j \\
                    a_i \otimes a_j & i \geq j 
                \end{cases}
            \end{align*}
            It is defined on a basis and extends uniquely to all elements.

            Surjectivity:
            Pick any $a_i \otimes a_j$. If $i \geq j$, then $\alpha(a_i \otimes
            a_j) = a_i \otimes a_j$, otherwise
            \begin{equation*}
                \alpha(a_j \otimes a_i + a_i \otimes a_j) = a_i \otimes a_j.
            \end{equation*}
            So the image contains all basis elements, which proves surjectivity.

            Injectivity:
            \begin{align*}
                \sum^n_{i=1} \sum^n_{j=1} b_{ij}\ \alpha(a_i \otimes a_j) &=
                \sum^n_{i=1} \left(\sum^i_{j=1} b_{ij}\ a_i \otimes a_j +
                \sum^n_{j=n+1} b_{ij}\ ( a_i \otimes a_j - a_j \otimes a_i
                )\right) \\
                &= \sum^n_{i=1} \left(\sum^{i - 1}_{j=1} (b_{ij} - b_{ji})\ a_i
                \otimes a_j + b_{ii} a_i \otimes a_i + \sum^n_{j=n+1} b_{ij}\
                a_i \otimes a_j \right)
            \end{align*}
            So using that $a_i \otimes a_j$ form a basis we deduce that $b_{ij}
            = 0$ for $i \leq j$. And for $i > j$ we have $b_{ij} = b_{ji}$ which
            gives $b_{ij} = 0$ for those $i, j$ as well.

            So $\alpha$ maps basis $\{ a_i \otimes a_j \mid i, j \in \{1,
            \ldots, n\}\}$ to images of these elements. Denote this new basis
            with $S' = \{ \alpha(a) \mid a \in S\}$.

            We observe that $K$ is the submodule generated by basis elements (of
            our new basis) $a_i \otimes a_j - a_j \otimes a_i \in S'$. There are
            $\frac{n (n - 1)}{2}$ of these elements

            So $\alpha \colon A^n \otimes_A A^n \rightarrow A^n \otimes_A A^n$
            restricts to an isomorphism
            \begin{equation*}
                \Sym^2_A(A^n) = (A^n \otimes_A A^n) / K \longrightarrow A^{n \times
                n} / A^{\frac{n (n - 1)}{2}}.
            \end{equation*}
            The rank of module on the left is $n^2 - \frac{n (n - 1)}{2} =
            \frac{n (n + 1)}{2}$, so that must also be the rank of
            $\Sym^2_A(A^n)$.

            For the case of $\Lambda^2_A(M)$ we do not have to shift the basis.
            If we quotient $A^n \otimes_A A^n$ by the submodule generated by
            $\{a_i \otimes a_i \mid i \in \{1, \ldots, n\}\}$ the tensor product
            becomes antisymmetric:
            \begin{equation*}
                0 = (a_i + a_j) \otimes (a_i + a_j) = a_i \otimes a_j + a_j
                  \otimes a_i.
            \end{equation*}
            So the basis of $\Lambda^2_A(M)$ is $\{ a_i \otimes a_j \mid i < j
            \}$. Simply counting the basis gives us the rank $\frac{n (n -
            1)}{2}$.



    \end{enumerate}
\end{exercise}

\end{document}

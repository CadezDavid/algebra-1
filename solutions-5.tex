\newcommand{\sheet}{5}
\documentclass{article}

\usepackage[english, german]{babel}
\usepackage{amsthm,amssymb,amsmath,mathrsfs}
\usepackage[shortlabels]{enumitem}
% \usepackage[tmargin=1.25in,bmargin=1.25in,lmargin=1.2in,rmargin=1.2in]{geometry}


\newcommand{\C}{\mathbb{C}}
\newcommand{\R}{\mathbb{R}}
\newcommand{\N}{\mathbb{N}}
\newcommand{\Q}{\mathbb{Q}}
\newcommand{\Z}{\mathbb{Z}}

\DeclareMathOperator{\id}{id}
\DeclareMathOperator{\im}{im}
\DeclareMathOperator{\GL}{GL}
\DeclareMathOperator{\sgn}{sgn}
\DeclareMathOperator{\Tor}{Tor}

\newenvironment{exercise}[1] {
  \vspace{0.5cm}
  \noindent \textbf{Exercise~{#1}.}
} {
  \vspace{0.5cm}
}
\newenvironment{claim} {
  \noindent \textbf{Claim.}
} {
}

\title{Algebra 1\\Exercise sheet \sheet}
\author{Solutions by: Eric Rudolph and David Čadež}

\date{\today}


\begin{document}

\maketitle

\begin{exercise}{1}
    Pick an ascending chain of ideals in $A$
    \begin{equation*}
        I_0 \subseteq I_1 \subseteq I_2 \subseteq \cdots.
    \end{equation*}
    For every $j = 1, \ldots, n$ define $p_j \colon A \rightarrow A / a_j$ and
    then $p_j(I_i) \subseteq A / a_j$ and observe the chains
    \begin{equation*}
        p_j(I_0) \subseteq p_j(I_1) \subseteq p_j(I_2) \subseteq \cdots.
    \end{equation*}
    Since they all terminate, pick $M$ to be the index when the longest one
    terminates. Suppose $0 \not= a \in I_{M+1} \setminus I_M$. Because the
    intersection of $a_j$ is $\{0\}$, we have $a \notin a_j$ for some $j$. Fix
    that $j$. That means $p_j(a) \not= 0$ for that $j$. That means $p_j(a) \in
    p_j(I_{M+1}) \setminus p_j(I_M)$, which is contradiction.

    Since $p_j(I_{M+1}) \setminus p_j(I_M) = \emptyset$, we have $a \in p_j(I_M)$
    for every $j$. That means $a = b_j + c_j$ for $b_j \in I_M$ and $c_j \in
    a_j$. Look at
    \begin{equation*}
        a^n = (b_1 + c_1) \cdots (b_n + c_n) \in I_M
    \end{equation*}
    \begin{equation*}

    \end{equation*}


\end{exercise}


\end{document}

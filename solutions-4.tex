\newcommand{\sheet}{3}
\documentclass{article}

\usepackage[english, german]{babel}
\usepackage{amsthm,amssymb,amsmath,mathrsfs}
\usepackage[shortlabels]{enumitem}
\usepackage{tikz}
\usepackage{tikz-cd}

% \usepackage[tmargin=1.25in,bmargin=1.25in,lmargin=1.2in,rmargin=1.2in]{geometry}


\newcommand{\C}{\mathbb{C}}
\newcommand{\R}{\mathbb{R}}
\newcommand{\N}{\mathbb{N}}
\newcommand{\Q}{\mathbb{Q}}
\newcommand{\Z}{\mathbb{Z}}

\DeclareMathOperator{\id}{id}
\DeclareMathOperator{\im}{im}
\DeclareMathOperator{\GL}{GL}
\DeclareMathOperator{\sgn}{sgn}
\DeclareMathOperator{\Tor}{Tor}
\DeclareMathOperator{\Sym}{Sym}

\newenvironment{exercise}[1] {
  \vspace{0.5cm}
  \noindent \textbf{Exercise~{#1}.}
} {
  \vspace{0.5cm}
}
\newenvironment{claim} {
  \noindent \textbf{Claim.}
} {
}

\title{Algebra 1\\Exercise sheet \sheet}
\author{Solutions by: Eric Rudolph and David Čadež}

\date{\today}


\begin{document}

\maketitle

\begin{exercise}{1}
    \begin{enumerate}
        \item So ${\{f_n\}}_n \in A[[T]]$ is a sequence of elements such that $f_n
            \in {(T)}^n$. Then we can define $f = \sum^{\infty}_{n = 0} f_n$,
            because every coefficient will have only finitely many summands.
            Then we have $f - \sum^{n}_{k = 0} f_k \in {(T)}^{n+1}$ by
            the definition of $f$. Also, if there would be $g \in A[[T]]$ which
            is not equal to $f$ at the coefficient at degree $m$, then $g -
            \sum^{m}_{k = 0} f_k \notin {(T)}^{m+1}$.
        \item Suppose $A$ is noetherian and $I \subseteq A[[T]]$ ideal. At the
            lectures we have shown that in this case $A[T]$ is noetherian, which
            we will use.
            Define the set
            \begin{equation*}
                B = \{ a T^n \mid \exists f \in I \colon f = a T^n +
                \text{higher terms} \}
            \end{equation*}
            and the ideal it generates
            \begin{equation*}
                J = (B) \subseteq A[T].
            \end{equation*}

            \begin{claim}
                For any generating set $B$ of an ideal $J$ in a noetherian ring
                $A$, we can find a finite subset $C \subseteq B$ that generates
                this ideal $J$.
            \end{claim}

            \begin{proof}
                Set $C = \{c\}$ for any $c \in B$. If $(C) = J$, we are done.
                Otherwise pick any $a \in J \setminus (C)$ and write $a =
                \sum^{n}_{i=1} a_i c_i$. Since $a \notin (C)$, there is $c_i
                \notin C$ and we can add all these $c_i$ to $C$. In such a step
                we increase $(C)$ by at least $a$. We also add finitely many
                elements in every step. Since the ring is noetherian, this
                process terminates and we get a finite set $C \subseteq B$ that
                generates $J$.
            \end{proof}

            Since $A[T]$ is neotherian and $B$ generates $J$, we can find $\{
            a_1 T^{n_1}, \ldots, a_k T^{n_k} \} \subseteq B$ such that $J =
            (a_1 T^{n_1}, \ldots, a_k T^{n_k})$. 

            Denote with $f_i = a_i T^{n_i} + \text{higher terms} \in
            I$ suitable $f_i \in I$. Pick any $g = b T^m + \text{higher terms}
            \in I$. Then we have $c^1_m, \ldots, c^k_m \in A[T]$ such that
            $\sum^k_{i=1} c^i_m a_i T^{n_i} = b T^m$. Observe that $c^i_m$ can
            be taken to be monomials (powers $\not= m$ cancel anyway). So
            \begin{equation*}
                g - \sum^k_{i=1} c^i_m f_i \in I \cap {(T)}^{m+1}.
            \end{equation*}
            Continuing so forth we get power series $g_1, \ldots, g_k \in
            A[[T]]$ defined by
            \begin{equation*}
                g_i = c^i_m T^m + c^i_{m+1} T^{m+1} + c^i_{m+2} T^{m+2} \cdots
            \end{equation*}
            Note that $c^i_m \in A[T]$, so this does not present the
            coefficients exactly. But such $g_i$ exists and is unique, which we
            can show using first part of the exercise. By construction we have
            \begin{equation*}
                g = \sum^k_{i=1} g_i f_i,
            \end{equation*}
            which shows $I = (f_1, \ldots, f_k)$.
    \end{enumerate}
\end{exercise}

\begin{exercise}{2}
    \begin{enumerate}
        \item First observe that every element $f \in \C[[z]]$ with non-zero
            constant term is invertible in the ring of power series $\C[[z]]$
            with positive radius of convergence. We know it is invertible as a
            formal power series from the lectures. The radius of convergence is
            positive, since the inverse $\frac{1}{f}$ is bounded on some small
            neighbourhood around $0$ (follows simply from continuity of $f$). So
            there is a ball around $0$ where $\frac{1}{f}$ does not have
            singularities and because the radius of convergence is the distance
            to the nearest singularity, it is positive.

            Let $I \subseteq \C[[z]]$ be an ideal. Pick $f \in I$. If constant
            term of $f$ is non-zero, then $I = (1)$. Otherwise there exists $k
            \in \N$ such that $f = z^k g$, where $g$ is a unit. This $k$ is just
            the position of the first non-zero coefficient. So $z^k \in I$.

            So $I$ is clearly defined by the minimum position of non-zero
            coefficient over all elements $f \in I$. Let $l \in \N$ be such.
            Then for every $h \in I$ either $h = z^l g$ for some $g \in \C[[z]]$
            or $h = z^m g$ for some unit $g$. In first case we have $h \in
            (z^l)$ and in the other contradiction with the minimality of $l$. So
            $I = (z^l)$. So $\C[[z]]$ seems to even be a PID.
        \item We claim $(\sin(x)) \subsetneq (\sin(\frac{x}{2})) \subsetneq
            (\sin(\frac{x}{4})) \subsetneq (\sin(\frac{x}{8})) \subsetneq
            \cdots$ is an infinite chain that does not terminate. Inclusions
            follow from equation
            \begin{equation*}
                \sin(\frac{x}{2^{n+1}}) \cos(\frac{x}{2^{n+1}}) = \sin(\frac{x}{2^n}).
            \end{equation*}
            And they are strict, because
            \begin{equation*}
                \frac{\sin(\frac{x}{2^{n+1}})}{\sin(\frac{x}{2^n})}
            \end{equation*}
            is not a holomorphic function, it has a pole at $2^n \pi$.
    \end{enumerate}
\end{exercise}


\begin{exercise}{3}
    \begin{enumerate}
        \item 
        \item
    \end{enumerate}
\end{exercise}

\begin{exercise}{4}
    Let $A$ be a PID.
    \begin{enumerate}
        \item Let $a \in A \setminus \{0\}$ and $\pi \in A$ prime.

            Lets first suppose that $\pi^{n+1} \nmid a$ and show
            $\dim_{A/\pi} \pi^n B / \pi^{n+1} B = 0$.

            Pick $\pi^n b + (a) \in \pi^n B$. Since $\gcd(\pi^{n+1}, a) =
            \pi^n$, we have $\alpha, \beta \in A$ such that $\alpha \pi^{n+1} +
            \beta a = \pi^n$. So $\alpha b \pi^{n+1} + \beta b a = b \pi^n$ and
            since $\beta b a \in (a)$ we have $b \pi^n + (a) = \alpha b
            \pi^{n+1} + (a) \in \pi^{n+1} B$. This proves $\pi^{n+1} B \subseteq
            \pi^n B$, which we had to show.

            Lets suppose now $\pi^{n+1} \mid a$. Write $a = u \pi^{n+1}$. We
            claim $\pi^n + (a) \in \pi^n B \setminus \pi^{n+1} B$. That is true,
            because $\pi^n + x a = \pi^n + x u \pi^{n+1}$ is not divisible by
            $\pi^{n+1}$ for any $x \in A$. So we have found a non-trivial
            element in the vector space $\pi^n B / \pi^{n+1} B$ and the
            dimension must be at least $1$. To show it is exactly $1$ we can
            show that every two elements are linearly dependent. Pick $\pi^n b +
            (a), \pi^n c + (a) \in \pi^n B$. If $\pi \mid b$ or $\pi \mid c$,
            then one of the vector is $0$ and they are linearly dependent.
            Otherwise pick $\alpha + (\pi), \beta + (\pi) \in A/\pi$ such that
            $\alpha b + (\pi) + \beta c + (\pi) = 0 + (\pi)$ (here we use
            $\gcd{}$ and the fact that $A$ is a PID again). Then $\alpha \pi^n b
            + \beta \pi^n c + (a) = \pi^n (\alpha b + \beta c) + (a) \in
            \pi^{n+1} B$ and thus a zero vector.
        \item Suppose $M = A^r \oplus A / a_1 \oplus \cdots \oplus A / a_k$, $N
            = A^s \oplus A / b_1 \oplus \cdots \oplus A / b_l$ with $a_1,
            \ldots, a_k, b_1, \ldots, b_l \in A$ non-zero and $a_1 \mid a_2 \mid
            \ldots \mid a_k, b_1 \mid b_2 \mid \ldots \mid b_l$. Suppose also $M
            \cong N$ as $A$-modules.

            If we have an isomorvarphism $\varphi \colon M \rightarrow N$, we can
            easily show $\Tor(M) \cong \Tor(N)$. Then we can quotient with these
            torsion parts and get isomorvarphism between free modules, which are
            isomorvarphic exactly when their ranks are the same. So we have $r =
            s$. Also, since $\varphi$ maps torsion elements to torsion elements, we
            can remove free parts of both $M$ and $N$.

            For every $x \in A/a_1 \oplus \cdots \oplus A/a_k$ we have $a_1 x =
            0$ and thus $a_1 \varphi(x) = 0$. Because $\varphi$ is surjective,
            we get $a_1 \mid b_1$. By the same argument, only using
            $\varphi^{-1}$ we get $b_1 \mid a_1$. So $a_1 = u_1 b_1$ for some
            unit $u_1 \in A$. Suppose now $a_2 \not= u_2 b_2$ for any unit $u_2
            \in A$. WLOG $b_2 \nmid a_2$. Then $b_2 \varphi(1) \in A/b_1
            \subseteq N$ where $1 \in A/a_2 \subseteq M$. Since $0 \not= b_2 1
            \in A/a_2$, we have $0 \not= \varphi(b_2 1) \in A/b_1$ and so \dots
            We somehow continue this process and maybe show what we need to.
    \end{enumerate}
\end{exercise}

\end{document}
